\documentclass{article}
\usepackage[utf8]{inputenc}
\usepackage[T1]{fontenc}
\usepackage[spanish,french]{babel}
\usepackage{hyperref}

\pdfgentounicode=1

\hypersetup{
    colorlinks=true,
    linkcolor=red,
    filecolor=red,      
    urlcolor=red,
    pdftitle={Ayudame a mejorar el mundo o empecemos por creer que podemos},
    pdfpagemode=FullScreen,
    }

\title{Ayúdame a mejorar el mundo o empecemos por creer que podemos}
\author{Emmanuel Gomez Ospina}
\date{Junio 1 de 2k22}

\begin{document}

\maketitle


\section*{Dedicatoria}
Este escrito lo dedico a mi familia, y especialmente a mis padres, que son las personas que me han enseñado a creer, en mí mismo, en el bien de los demás y en los sueños. Por otro lado, también dedico esto a la sociedad del futuro, de la cual será parte mi hermano, pues quiero presentarles la posibilidad de que seamos conscientes de nuestra situación privilegiada y normalizar que a pesar de ser privilegiados socio-económicamente, y espacio-temporalmente, la salud mental, en declive en los tiempos contemporáneos, sea principal foco de nuestras energías, porque al ser seres más adaptados a un estilo de vida moderno, con menos urgencia por resolver las necesidades básicas, sufrimos el dilema existencial, y puede no permitirnos desarrollar profundamente nuestras habilidades, y como consecuencia última, puede que desistamos de pelear por la vida, porque como decía Séneca el Joven, \textit{"A veces, incluso el vivir es un acto de valentía"}, evitando que siquiera desarrollemos ideales por los cuales pelear, porque no creo que tengamos ideales ni sueños sin haber tenido contacto real con nuestro entorno, contacto conflictivo que permite desarrollar una personalidad, porque no tenemos una personalidad, sino que la formamos con lo que percibimos del mundo y nos \href{https://youtu.be/-cpKC1II1Dc}{apasiona} de él.


\section{Introducción}
En el libro intento exponer mi perspectiva sobre ideas como la mayoría de edad de \textit{Immanuel Kant}, el porqué creo que ésta no es una sola en la sociedad moderna, y la necesidad de superar varias de las etapas y reconocerlas para no quedar atascados en etapas intermedias. \\
%para luego, las ideas son:
%básicamente está la mayoría de edad de kant, que se refiere a que la persona es capaz de razonar por sí misma y ser consciente de su situación. Mi problema con este concepto, es que en lo personal, pienso que un individuo necesita varias etapas para entenderse a sí mismo, y al mismo tiempo, éstas etapas están relacionadas con la situación espacio-temporal de la persona.
% en lo personal, yo he experimentado 2 etapas de mayoría de edad, o puede que sean más, pero para mí, podría diferenciar el cuándo empecé a ser consciente de mi mismo, pero en ese momento no fue tan profundo como cuando en la pandemia empezó mi siguiente transformación principalmente se centró al rededor del existencialismo pues tenía muchas dudas después de haber tenido una "carrera pasada" llena de seguridades (sabía que quería estudiar, qué quería hacer de tesis, que quería hacer después de graduarme) pero al darme cuenta que mis ideas estaban caducando por una u otra razón, tuve una fuerte depresión que surgió por mi falta de control a la hora de hacer lo que me proponía, en parte porque no veía el porqué necesitaba poner tanta energía en cosas cotidianas si no tenía idea de para dónde iba. La siguiente parte de la transformación fue entender parte de mis obligaciones y responsabilidades, y entender cómo mi forma de ser afectaba a los demás, y el hecho de viajar entre ciudades distintas con objetivo de estudio y vacaciones me ayudó a solidificar mi postura frente a la noción de responsabilidad.

También me gustaría hablar de mi relación con el estoicismo, cómo ha sido un concepto de gran ayuda en mi madurez y mi estancia en Medellín en el presente. \\

Me gustaría exponer mi perspectiva sobre cómo abordaría parte de los problemas principales de la civilización en la que vivo, intentando dar argumentos, a pesar de que la profundidad en la que puedo ir en cada uno de ellos es limitada debido al tamaño del libro, me gustaría exponer algunos flujos de pensamiento que utilizo normalmente para aproximarme a los problemas desde un punto de vista simple, basado en el principio de humildad, el hecho de tener presente que no soy experto en el tema, pero busco formar un argumento lo suficientemente sólido, para poder creer en él y defenderlo ante los posibles puntos negativos que genere, y estar abierto a mejorarlo o descartarlo por medio de la duda razonable del método científico, pero con la excepción de que se debe tener en mente de que no hay solución perfecta en la inmediatez, mas sin embargo, puede que muchas estrategias guíen a una convergencia que dé como resultado un estado similar después de que éstas tomen su debida evolución, tomando cantidades de tiempo, recursos, políticas, emociones y corrientes de pensamiento distintas. Por lo tanto es importante analizar las distintas variables, y tener en cuenta que muchas de los problemas relacionados con el gobierno (Colombiano en mi caso) tienen una componente de efusividad en el mediano plazo (4 años por gobierno) lo cual hace que parte de mis soluciones tengan un lado anárquico y más basado en lo personal, aceptando errores del sistema (influenciada por el estoicismo). \\

Consiguiente a esto también expongo mis ideas sobre el porqué es importante que cada uno de nosotros busquemos mejorar nuestra civilización de origen, hablar sobre la gratitud, el sentido de pertenencia, y analizar en profundidad la frase "Los tiempos duros crean hombres fuertes; los hombres fuertes crean buenos tiempos; los buenos tiempos crean hombres débiles; los hombres débiles crean tiempos duros".

Otro tema del cual quisiera hablar es de cómo el cambio es el único ritmo constante en el cosmos y el porqué debemos rechazar constantemente nuestra actitud hacia nuestro instinto más moderno de bienestar estacionario y buscar reencontrar nuestra naturaleza \textit{nómada} un poco más antigua, pero más en paz con la idea de adaptación y que es un resultado más tangible de la evolución natural a grandes escalas de tiempo y civilizaciones.



\section{Sobre la educación en Colombia y su acercamiento al sector rural}

Hoy, viendo el
\href{https://youtu.be/kicLOUbQ53g}{siguiente video}, comenté lo siguiente:\\

Exactamente, así mismo como la inversión en el ejército ha sido prolongada y se recibe apoyo de los EE. UU.\footnote{Se refiere a "estado" y "unido" en plural, por eso la E y la U están repetidas. No lo sabía hasta hace poco, es una regla del español. Otro ejemplo son Juegos Olímpicos, JJ. OO.},
también se debe invertir en cuestión de educación a largo plazo, con fuerza y desde la educación básica, porque en la etapa universitaria ya la gente que está, llegó filtrada por el sistema actual con tan pocas oportunidades, especialmente en sectores apartados y socialmente desfavorables para los niños. En Bogotá la gente sí protesta porque no tienen ruta de Transmilenio para ir al colegio mientras que las personas del campo se matan 2h ida 2h venida para estudiar, y no les explican nada que les sirva. Lo cierto es que muchas de estas soluciones no es fácil que vengan por parte del estado, toca que tengamos sentido de pertenencia, y los que hemos sido afortunados, devolvamos el favor, haciendo aportes sociales, no necesariamente con dinero, porque es de las cosas que peor se administra, sino con proyectos sociales y educativos en busca de desarrollo. Uno mismo podría ir y dar parte de su conocimiento teórico o líquido en materias y temas que puedan ser de ayuda, falta es que tengamos más sentido de pertenencia por aquel estado que por mal que esté, nos dió la educación pública, un aporte colectivo de los impuestos de los demás, plata que pudo generar cambio y no fue absorbida por la corrupción, plata que uno mismo me puede ejemplificar como buena inversión.\\

A esto, \textbf{@\textit{Cosas del mundó}} respondió: No. Lo que se debe hacer es privatizar la educación estatal. \\

Luego le respondí: Si bro, es parte del argumento, aunque es sostenible el hecho de privatizar, me gusta también que se pueda verlo desde una perspectiva humanitaria. No descarto que seguramente el modelo sea más sostenible de forma privada, porque al final es más fácil lograr un propósito con plata, porque eso mueve el mundo, pero como te digo, los que no la tenemos o no la suficiente, podríamos empezar a hacer un cambio, especialmente con la venida de nuevas tecnologías al campo. Un saludo y gracias por leer.

Este segmento fue escrito entre el día 1 de Junio y el 2 de junio de 2k22.




\end{document}
